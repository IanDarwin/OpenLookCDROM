\section{CSTR Equations}
The equations for the \underline{C}ontinuous flow \underline{S}tirred
\underline{T}ank \underline{R}eactor (CSTR) is a well-known model equation 
from chemical engineering which describe the rate of reactant
conversion in a non-isothermal mixture.  Here we present the equations
for two such coupled cells, described by Kubi\u{c}ek \cite{kubicek1}:

\begin{eqnarray*}
   \dot{x}_1 & = & (1-a) x_3 - x_1 + b(1-x_1) E(x_2; d) \\
   \dot{x}_2 & = & (1-a) x_4 - x_2 + b e (1-x_1) E(x_2;d) - f(x_2-h) \\
   \dot{x}_3 & = & \frac{\normalsize \mbox{$b$}}{\normalsize \mbox{$c$}} \left [
		   x_1 - x_3 + c(1-x_3) E(x_4;d) \right ] \\
   \dot{x}_4 & = & \frac{\normalsize \mbox{$b$}}{\normalsize \mbox{$c$}} \left [
		   x_2 - x_4 + ce(1-x_3) E(x_4;d) - g (x_4-h) \right ] \\
\end{eqnarray*}
where the function
\[
         E(y;p) = \exp\left (  { {\normalsize \mbox{$y$}} \over 
				 {\normalsize \mbox{$1 + y/p$}} }\right )
\]
expresses the relation between the reaction rates and normalized
temperature.  If the parameter $a$ is set equal to one, the cells are
decoupled and the system may be used to study the two-reactant version
for the variables $(x_1,x_2)$ as given by Uppal, et. al \cite{uppal1}.
See Guckenheimer \cite{gucken1} and Golubitsky and Schaeffer \cite[pp.
16-25]{golubitsky3} for bifurcation results in this case.
\medskip

\noindent The file {\bf cstr1} contains a {\bf DsTool} configuration file which
displays Hopf bifurcation curves in
the $(b,c)$ plane, separating this two-parameter space into regions of
different behavior.  These data may be compared with that reported in 
\cite{kubicek1}.

