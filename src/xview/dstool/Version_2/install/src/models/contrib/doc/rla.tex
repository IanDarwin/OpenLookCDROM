\section{Model of a compartmental neuron}

This six dimensional vector field is an elaboration of the classical
Hodgkin-Huxley equations. It is designed to described the behavior
of a particular neuron, the Anterior Burster (AB) cell of the 
lobster stomatogastric ganglion. This cell is a conditional burster
that serves as a pacemaker for the pyloric circuit within this ganglion.
The model contains a representation of the membrane potential (v),
internal, free calcium concentration (Ca), and the currents carried
by several ion specific channels. There are sodium, potassium and
"leak" currents like those of the Hodgkin-Huxley equations  together
with a calcium current and two additional potassium currents: a
calcium activated potassium current and a transient A current. Without
the A current, the model is an adaptation one used by Rinzel and Lee
to produce a model for bursting behavior of the Aplysia R15 neuron.

The data file {\tt rla.dat} contains parameter space data for variations
of the conductances of the calcium activated potassium and transient A
currents. There are curves (originally computed with Maple) that give
saddle-node and Hopf bifurcations. The large region bounded by the 
Hopf bifurcation curve approximately delineates the parameter space
region in which bursting behavior is found. The additional data is the 
result of an investigation of how the bursting region is divided into
regions in which there are a fixed number of action  potentials of a
bursting pattern. The symbols other than squares show parameter values
at which more complex behavior was located. There is a large region of
chaotic behavior near the bottom of the bursting region, inside of which
there seem to be homoclinic orbits of Silnikov type. The phase space 2-D 
window is set to display calcium concentration versus membrane potential.
We refer to \cite{epsteinmarder,GGHW,rinzel1} for more information.

The equations are

\begin{eqnarray} 
C_{m}\dot{v} = -(
\underbrace{g_{Na}m^3h(v-v_{Na})}_{I_{Na}} + 
\underbrace{g_{Ca} \frac{z}{0.5+c} (v-v_{Ca})}_{I_{Ca}} + \nonumber \\
\underbrace{g_Kn^4(v-v_K)}_{I_{K}} + 
\underbrace{g_{KCa}\frac{c}{0.5+c}(v-v_{K})}_{I_{KCa}} + \nonumber \\
\underbrace{g_{A}m_A^3h_A(v-v_K)}_{I_{A}} + 
\underbrace{g_l(v-v_l)}_{I_{l}}) + I_{ext}        \label{vdot}
\end{eqnarray} 

\begin{equation} 
\dot{c}  = \rho(\frac{(k_{Ca}z(v_{Ca}-v)}{(1+2c)} - c)   \label{cdot}
\end{equation} 

\begin{equation} 
\dot{n}  = \lambda_n (a_n(1-n) - b_n n)    \label{ndot}
\end{equation} 

\begin{equation} 
\dot{h}  =  \lambda_h (a_h(1-h) - b_h h)   \label{hdot}
\end{equation} 

\begin{equation} 
\dot{z} = (z_v - z)/ \tau_{z}    \label{zdot}
\end{equation} 

\begin{equation} 
\dot{h}_A  =  (h_{Ai} - h_A) k_A   \label{hadot}
\end{equation} 

\begin{equation}
a_m = \frac{\frac{127}{105}v + \frac{201}{7}}{10-10 
     e^{-\frac{201}{70}-\frac{127}{1050}v}}    \label{am} \\
\end{equation}
 
\begin{equation}
b_m = 4e^{-\frac{188}{63}-\frac{127}{1890}v} \label{bm}  \\
\end{equation}

\begin{equation}
m = \frac{a_m}{a_m+b_m}  \label{m}  \\
\end{equation}

\begin{equation}
a_h = \frac{7}{100} e^{-\frac{94}{35}-\frac{127}{2100}v}   \label{ah} \\
\end{equation}

\begin{equation}
b_h = \frac{1}{1+e^{-\frac{83}{35}-\frac{127}{1050}v}}  \label{bh}  \\
\end{equation}

\begin{equation}
a_n = \frac{\frac{127}{105}v+\frac{166}{7}}{100-100 
     e^{-\frac{83}{35}-\frac{127}{1050}v}}  \label{an}  \\
\end{equation}

\begin{equation}
b_n = \frac{1}{8} e^{-\frac{59}{140}-\frac{127}{8400}v}  \label{bn} \\
\end{equation}

\begin{equation}
m_A = \frac{1}{1+e^{\frac{v-v_a}{s_a}}}  \label{ma} \\
\end{equation}

\begin{equation}
h_{Ai} = \frac{1}{1+e^{\frac{v-v_b}{s_b}}}   \label{ha} \\
\end{equation}

\begin{equation}
z_v = \frac{1}{1+e^{-\frac{15}{100}(v - z_b)}} \label{zv} \\
\end{equation}

The following parameters are fixed in the computations

\begin{tabular}{||l|l|r||}               \hline  \hline
PARAMETER &   VALUE  &     UNITS  \\   \hline  \hline
$\rho$      &    $0.003$ &       $ms^{-1}$    \\ 
$\lambda_n$ &    $0.8$ &       $ms^{-1}$    \\
$\lambda_h$ &    $0.8$ &       $ms^{-1}$    \\
$k_{A}$     &    $140$ &       $ms^{-1}$    \\ \hline
$\tau_z$    &    $23.5$ &       $ms$    \\\hline
$k_{Ca}$     &    $0.0078$ &       $mV^{-1}$    \\  \hline
$z_b$       &    $-50$ &       $mV$   \\
$v_a$       &    $-12$ &       $mV$    \\
$v_b$       &    $-62$ &       $mV$    \\
$s_a$       &    $-26$ &       $mV$    \\
$s_b$       &    $16$ &       $mV$    \\   
$v_{Na}$    &    $30$ &       $mV$    \\
$v_{Ca}$    &    $140$ &       $mV$    \\
$v_K$       &    $-75$ &       $mV$    \\
$v_l$       &    $-40$ &       $mV$    \\      \hline
$g_l$       &    $0.0854$ &       $\mu S$    \\      
$g_{Ca}$       &    $0.04$ &       $\mu S$    \\      
$g_{Na}$       &    $15$ &       $\mu S$    \\      \hline
$C_m$       &    $1$ &       $nF$    \\      \hline  \hline
\end{tabular}

