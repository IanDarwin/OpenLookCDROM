\section{The Field-Worfolk Equations}

Let $\bar{G}$ be the group generated by the matrices
\[ \rho = \left( \begin{array}{cccc}
-1&0&0&0\\ 0&1&0&0 \\ 0&0&1&0 \\ 0&0&0&1
\end{array} \right),  \; \; \;
%\]
%and 
%\[ 
\sigma = \left( \begin{array}{cccc}
0&1&0&0 \\ 0&0&1&0 \\ 0&0&0&1 \\ 1&0&0&0
\end{array} \right). \]
The subgroup $G \subset \bar{G}$ consists of those matrices in $\bar{G}$
that preserve orientation ({\it i.e.}, have positive determinant).  The group
$G$ has $32$ elements and is given by products of $\rho$ and $\sigma$ with an 
even number of terms.

The Taylor
series of degree $3$ for a family of $G$-equivariant vector fields
bifurcating at the trivial equilibrium has the form
\begin{eqnarray*}
\dot{x} &=& (l+ar^2+by^2+cz^2+dw^2)x+eyzw \\
\dot{y} &=& (l+ar^2+bz^2+cw^2+dx^2)y-exzw \\
\dot{z} &=& (l+ar^2+bw^2+cx^2+dy^2)z+exyw \\
\dot{w} &=& (l+ar^2+bx^2+cy^2+dz^2)w-exyz 
\end{eqnarray*}
with $$r^2 = x^2+y^2+z^2+w^2.$$ Here $(x,y,z,w) \in
R^4$ and $l,a,b,c,d,e$ are parameters.
% with $l$ the parameter which induces bifurcation at the origin.  
This system of differential contains a wealth of interesting dynamical
behavior.  The most interesting feature is a direct bifurcation
from an equilibrium point to a chaotic attractor as described by
Guckenheimer and Worfolk \cite{GuckenheimerWorfolk:instant}.

The following files are supplied with this model system.  We
suggest that you use a variable stepsize integrator to examine this
system (e.g., Runge Kutta 4QC or Bulirsch-Stoer).
\begin{enumerate}
\item {\tt fw1.dat}
For these parameter values the trajectory will wander around the invariant curves
making up a heteroclinic cycle appearing to be asymptotic to a large periodic orbit.
\item {\tt fw2.dat}
For these parameter values, the heteroclinic cycles are broken by fixed points
lying on the old cycle.  The system now appears to have a complicated
attractor.
\item{\tt fw3.dat}
The asymmetric zeros are now involved in the attractor to give
a wealth of complicated behavior.  You will need to look at long trajectories
for these parameter values.
\end{enumerate}

