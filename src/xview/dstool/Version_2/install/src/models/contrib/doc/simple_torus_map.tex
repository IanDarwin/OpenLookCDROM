\section{Simple Torus Mapping}
        A mapping on the two-dimensional torus, ie, the unit square
with edges identified.  The mapping is defined by
\begin{eqnarray*}
	x  &\to& x+\epsilon(\Omega_x+\cos(2 \pi x)+ a \cos(2 \pi y)) \\
	y  &\to& y+\epsilon(\Omega_y+\sin(2 \pi x)+a \sin(2 \pi y))
\end{eqnarray*}

This mapping is a ``simple'' mapping in the sense that its {\em
resonance region} is as simple as possible.  We may think of this
particular resonance region as the
set of parameter values for which there exists a fixed point for the mapping.

Associated with this map is the data file {\tt simple1.dat}.  Loading this
file draws a resonance region in the $(\Omega_x, \Omega_y)$ parameter
space. There is a green annulus which represents the set of
saddle-node bifurcations.  Inside the annulus there exists a pair of
fixed points for the map.  The smaller ellipses inside of the ellipse
are points at which Hopf or Neutral Saddle bifurcations occur.  For
more information, consult \cite{BGKM} or \cite{BGKM:IMA}.