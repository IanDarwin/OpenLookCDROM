\section{Kim-Ostlund Torus Mapping}

A mapping on the two-dimensional torus, ie, the unit
square with edges identified.

The mapping is defined by a nonlinear perturbation of
a translation:
\begin{eqnarray*}
	x &\to& x + \Omega_x - a s/(2 \pi) \sin(2 \pi y) \\
        y &\to& y + \Omega_y - a/ (2 \pi s) \sin(2 \pi x)
\end{eqnarray*}
	 The variables  $x$  and  $y$   are periodic with period 1.
The parameters $\Omega_x$ and $\Omega_y$ determine the net rotation in
the horizontal and vertical directions, $a$ is the amplitude of the
nonlinear perturbation, and $s$ is a parameter which (if $s \neq 1$)
introduces an asymmetry in the equations.

Loading the file {\tt ko1.dat} will draw a {\em resonance region} in the
$(\Omega_x, \Omega_y)$ parameter space.  A resonance region may be
thought of as the set of parameter values for which there exists a
periodic orbit of a fixed period.  For this example, the period of
interest is period five.  The resonance region is the projection of a
torus.  The green curves correspond to points of saddle-node
bifurcation, the blue curves are curves of Hopf bifurcation, and the
orange curves are parameters for which there is a so-called neutral
saddle.  For more information on the structure of resonance regions
for torus maps and on the dynamics of the Kim-Ostlund torus map, see
\cite{BGKM}.
