\section{Damped Driven Pendulum}

        This vector field describes the motion of a damped
        and periodically forced pendulum.  
        The vector field is given by
\begin{eqnarray*}
	\dot{q} &=& p \\
	\dot{p} &=& -\delta p - \sin(q) + A \sin(\omega t) + c.
\end{eqnarray*}
Here $\delta$ is the coefficient of damping, $A$ is the amplitude of
forcing, $\omega$ is the frequency of forcing, and $c$ is the constant
torque applied to the pendulum.  

For the case of constant torque ($A=0$) there are two associated data
files.  The first is {\tt pendulum1.dat} which loads bifurcation curves
into the $(\delta, c)$ parameter plane.  The green curve is the set of
saddle-node bifurcations for the pendulum equations, whereas the red
curve corresponds to the set of parameters for which there exists a
saddle connection.  

The second file is {\tt pendulum2.dat} which shows four blue curves in the
$(\delta, c)$ parameter plane.  The curves are the locus of points for
which there exists an attracting periodic orbit of period 3 (leftmost
curve), 5, 7, and 10 (rightmost curve). 

The literature concerning the torqued pendulum is extensive.  We refer
to \cite{AVK,GH} and references therein.

The case of $A \neq 0$ is sufficiently complicated that we refer to 
\cite{GH,salamsastry} for details. 
The file {\tt pendulum3.dat} sets up a Poincar\'{e} section which can be
used to study the structure of a chaotic attractor.
We suggest that you change to a quality control integrator like
Runge-Kutta 4QC or Bulirsch-Stoer.  The file {\tt pendulum4.dat} has
several points marked in the phase space; these point are close to
periodic orbits for the Poincar\'e map (periods 3, 4, and 5).

