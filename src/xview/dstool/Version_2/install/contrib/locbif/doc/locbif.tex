
\documentstyle[psfig]{article}

% ----------------------------------------------------------------------------------
% File of macros and standard definitions
% ----------------------------------------------------------------------------------

% Definition for bitmap images to be included
%\def\sunbitmap#1#2#3#4{%
%\hfill \hbox to #1{%
%\vbox to #2{
%\vss
%\special{pre: sunbitmap #3 #40}
%\special{ps: sunbitmap #3 #40}
%}%
%}%
%\hfill
%}


% Basic LATEX macros
\newcommand{\be}{\begin{equation}}
\newcommand{\ee}{\end{equation}}
\newcommand{\ben}{\begin{displaymath}}
\newcommand{\een}{\end{displaymath}}
\newcommand{\qed}{
\begin{flushright}
{\mbox{$\Box$}}\end{flushright}\vspace{.25in}}

% set special fonts
\font\setfont=eufm10

% Standard math macros
\newtheorem{theorem}{Theorem}[section]
\newtheorem{corollary}{Corollary}[section]
\newtheorem{prop}{Proposition}[section]
\newcommand{\proof}{{\bf Proof:} }
\newcommand{\realf}{\mbox{{\bf R}$^3$}}
\newcommand{\plane}{\mbox{{\bf R}$^2$}}
\newcommand{\realn}{\mbox{{\bf R}$^n$}}
\newcommand{\gameq}{\mbox{$\Gamma$-equivariant }}
\newcommand{\dstool}{\mbox{\bf dstool}}
\newcommand{\pvf}{\mbox{$u^{\prime}=P^Q(u)$}}
\newcommand{\group}{\mbox{\setfont G}}
\newcommand{\property}{\mbox{\setfont P}}
%\newcommand{\group}{\mbox{G}}
%\newcommand{\property}{\mbox{P}}
\newcommand{\real}{{\mbox{I} \hspace{-.03in} {\bf R} }}


% set margins to 1 inch + specified quantity:
 
\oddsidemargin 0.25in
\topmargin 0.25in
\headheight 0.25in

\textwidth 6in
\headsep 0in
\textheight 8.25in

% ----------------------------------------------------------------------------------
% end of standard definitions 
% ----------------------------------------------------------------------------------

\begin{document}

\section{Panel Descriptions}

\subsection{Main Panel}
  
\begin{figure}[tpbh]
\begin{picture}(400,325)(0,0)
%\put(60,0){\sunbitmap{3.0in}{3.0in}{fig1.ras}{75} }
\put(60,0){\psfig{figure=fig1.ps,height=3.0in,width=3.0in}}
\end{picture}
\caption{ \label{co1} Main panel for the interface to the locbif continuation package.  }
\end{figure}

\begin{description}

\item[Window title:] locbif Main Panel
\item[Function:]  This panel allows the user to select panels which control the type
of bifurcation problem desired, select the continuation parameters and control the 
supplementary panels which contain algorithm settings.
\item[Description:] The locbif Main Panel is opened by selecting the {\bf locbif}
option from the Panels menu button located on the Command Window.
If unpinned, the panel vanishes when any button is selected.

\item[Panel items:]\mbox{}
	\begin{itemize}
		\item {\bf Forward:}  This button initiates continuation in the forward direction.
		      The term {\em forward} here has a conventional (increasing parameter) 
		      connotation, not a geometrical one.
                \item {\bf Continue:}  This button cause locbif to continue tracing the bifurcation
		      curve, beginning at the last computed point, in the direction specified at
		      the last computation.
                \item {\bf  Backward:}  This button initiates continuation in the backward direction.
		      The term {\em backward} here has a conventional (decreasing parameter) 
		      connotation, not a geometrical one.
                \item {\bf Mode:}  This button item is used to select the type of bifurcation problem
		      desired.  Valid options are supplied for the continuation of bifurcation curves
		      for equilibria, limit cycles of autonomous vector fields, periodic orbits of
		      non-autonomous ODE's ({\em not implemented}), and fixed points of iterated
		      mappings ({\em not implemented}).
                \item {\bf Pause: }  Selects interupt mode for locbif.  Computations initiated in
		      locbif can always be interrupted by typing the keystroke combination
		      [Cntl] C in the X11 or OpenWindows term window used to start the dstool program.  
		      In this case, control will be passed back to the locbif panel for further
		      input by the user.  However, the locbif code can also be instructed to pause
		      at various points during the continuation of a bifurcation curve.  If Pause
		      is turned {\em Off}, locbif will complete a full sequence of computed points and then
		      return control to the dstool panel interface.  If Pause is set to {\em Special}, 
		      locbif will pause each time a special bifurcation point is computed (point of
		      higher codimension or algorithm failure).  If the {\em Iterate} option is selected, 
		      locbif will pause after the computation of each iterate along the bifurcation curve.
                \item {\bf Cont Panel: }  Opens and brings to the foreground the locbif Continuation
		      Panel.
                \item {\bf Orbit Panel: }  Opens and brings to the foreground the locbif Orbit
		      Panel.
                \item {\bf Display Panel: }  Opens and brings to the foreground the locbif Display
		      Panel.
                \item {\bf Cur State: }  Opens and brings to the foreground the locbif Current State
		      Panel.
                \item {\bf Selected Parameters: }  This list item presents the phase space variables and
		      dynamical system parameters which may be selected for continuation.  The number of
		      entries from this list depends upon the type of continuation problem desired;  the
		      required number is shown in the Curve Type menu item described below.  
                \item {\bf Active Parameters: } This display-only message item shows the current list
		      of selected variables and parameters activated for continuation using the Selected
		      Parameters item described above.
                \item {\bf Curve Type: } The Curve Type menu item is used to select the type of continuation
		      curve to be computed.  Of course, a different list is presented for each of the four
		      different Mode selections described above.  Table \ref{tab1} and Table \ref{tab2} show the
		      valid option for equilibrium and limit cycle mode selections, respectively.
                \item {\bf Read File Interface: }  Locbif can be initialized by reading an interface file
		      in a standard format.  The Read File Interface button is used to read such a file
		      and update the dstool panels and data structures associated with the data stored in 
		      the file.  Notice that if the panel entries are changed {\em after} a locbif
		      interface file is read, the new panel entries take precedence over those originally
		      read from the interface file when locbif is executed.
                \item {\bf Directory: } Directory where locbif interface file is located.
                \item {\bf Filename: } Name of locbif interface file.
        \end{itemize}
\end{description}

\begin{table}[htbf]
\centering
\begin{tabular}{||c||c|c|l||} \hline
Index & \makebox[.4in]{ Label }& \makebox[.4in]{Params}  & \makebox[.7in]{Description}
 \\ \hline \hline
1 & Orbit &  0 & Orbit Propagation \\
2 & Curve &  1 & Equilibrium Curve with no Bifurcation \\
  &       &    & Analysis \\
3 & Equilibrium & 1 & Equilibrium Curve \\
4 & Fold  & 2 & Fold Points \\
5 & Hopf & 2 & Hopf Points \\
6 & Double Eigenvalue & 2 & Double Eigenvalues \\
7 & Double Zero & 3 & Takens-Bogdanov Curve \\
8 & Fold + Hopf & 3 & Gavrilov-Guckenheimer Curve \\
9 & Double Hopf & 3 & Two Pair imaginary Eigenvalues \\
10 & Cusp & 3 & Cusp Points \\
11 & Hopf + Lyapunov Zero & 3 & Degenerate Hopf Curve \\
12 & Fold + BTC & 3 & Fold Points with Broken Transversality \\
   &            &    & Condition \\
13 & Hopf + BTC & 3 & Hopf Points with Broken Transversality \\
   &            &   & Condition \\
14 & Double Zero + Cusp & 4 & Degenerate Takens-Bogdanov \\
15 & Hopf + Cusp & 4 & Degenerate Gavrilov-Guckenheimer \\
16 & Double Zero + BTC & 4 &  Takens-Bogdanov with Broken Transversality \\
   &                   &   &  Condition \\
17 & Fold + Hopf + BTC & 4 &  Gavrilov-Guckenheimer with Broken \\
   &                   &   &  Transversality Condition \\
\hline \end{tabular}
\caption{\label{tab1} List of Continuation Curve Options for the Equilibrium Mode Setting.}
\end{table}

\begin{table}[htbf]
\centering
\begin{tabular}{||c||c|c|l||} \hline
Index & \makebox[.4in]{ Label }& \makebox[.4in]{Params}  & \makebox[.7in]{Description}
 \\ \hline \hline
1 & Orbit &  0 & Orbit Propagation \\
2 & Curve &  2 & Equilibrium Curve with no Bifurcation \\
3 & Limit Cycle & 2 & Limit Cycle Points \\
4 & Fold  & 3 & Fold Points \\
5 & Hopf & 3 & Hopf Points \\
6 & Flip & 3 & Flip Points \\
7 & Double Multiplier & 3 & Pair of Floquet Multipliers \\
8 & Double Fold & 4 & Double Foldpoints  \\
9 & Double Flip & 4 & Double Flippoints \\
10 & Fold + Hopf & 4 & Fold Curve with Hopf Degeneracy \\
11 & Flip + Hopf & 4 & Flip Curve with Hopf Degeneracy \\
12 & Fold + Flip & 4 & Fold Points with Flip Degeneracy \\
13 & Cusp & 4 & Cusp Points \\
14 & Fold + BTC & 4 & Fold Points with Broken Transversality \\
   &                   &   &  Condition \\
15 & Hopf + BTC & 4 & Hopf Points with Broken Transversality \\
   &                   &   &  Condition \\
16 & Flip + BTC & 4 &  Flip Points with Broken Transversality \\
   &                   &   &  Condition \\
\hline \end{tabular}
\caption{\label{tab2} List of Continuation Curve Options for the Limit Cycle Mode Setting.}
\end{table}



\clearpage
\subsection{locbif Continuation Settings Panel}

\begin{figure}[tpbh]
\begin{picture}(400,210)(0,0)
%\put(60,0){\sunbitmap{3.0in}{3.0in}{fig2.ras}{75} }
\put(60,0){\psfig{figure=fig2.ps,height=3.0in,width=3.0in}}
\end{picture}
\caption{ \label{co2} Panel used to control the locbif continuation algorithm settings.}
\end{figure}

\begin{description}

\item[Window title:] locbif Continuation Panel
\item[Function:]  This panel allows the user to select values and settings used to control
the locbif continuation algorithms.
\item[Description:] The locbif Continuation Panel is opened by selecting the {\bf Cont Panel}
button located on the locbif Main Panel.  If unpinned, the panel vanishes when any button is selected.

\item[Panel items:]\mbox{}
	\begin{itemize}
		\item {\bf h0crv:} Initial continuation stepsize (default: 0.1) 
		\item {\bf hmxcrv:} Maximum allowed continuation stepsize (default: 1.0) 
		\item {\bf angcrv:} Desired angle (in degrees) between the tangent vector
		      at the current point and the secant vector passing through the last
		      and current computed points.  Used in stepsize control.  (default: 10.0)
		\item {\bf dhcrv:}  Difference stepsize from numerical computation of the
		      Jacobian of the augumented continuation equations. (default: 1.0e-7)
		\item {\bf dhjac:}  Difference stepsize for numerical computation of the 
		      Jacobian of the dynamical system equations. (default: 1.0e-7)
		\item {\bf epscrv:}  Tolerance for the Newton correction. (default: 1.0e-4 )
		\item {\bf epscrs:}  Tolerance for self-crossing point determination.  Zero 
		      value indicates supression of the self-crossing point location is
		      desired.
		      (default: 1.0e-3)
		\item {\bf epszer:}  Tolerance for determining zero's of bifurcation functions.  Zero
		      value indicates supression of the root location is desired.
		      (default: 1.0e-3)
		\item {\bf epsext:}  Tolerance for determining extremal values of parameters.  Zero
		      value indicates supression of the extremal location is desired.
		      (default: 1.0e-3)
		\item {\bf algcrv:}  Selection of stepsize control algorithm. 
		\item {\bf maxit:}  Maximum number of allowed Newton corrector iterations.
		      (default: 7)
		\item {\bf modit:}  Number of Newton correction steps taken where Jacobian is
		      reevaluated at the beginning of each step.
		      (default: 2)
		\item {\bf ipsrng:}  Indicates the active parameter for which the non-transversality
		condition is assumed to hold.  Parameter used in extremal point continuation.
		(default: 1)
        \end{itemize}
\end{description}

\clearpage
\subsubsection{locbif Orbit Settings Panel}

\begin{figure}[tpbh]
\begin{picture}(400,250)(0,0)
%\put(130,0){\sunbitmap{3.0in}{3.0in}{fig3.ras}{75} }
\put(130,0){\psfig{figure=fig3.ps,height=3.0in,width=3.0in}}
\end{picture}
\caption{ \label{co4} Panel used to control the locbif orbit propagation routines.}
\end{figure}


\begin{description}

\item[Window title:] locbif Orbit Panel
\item[Function:]  This panel allows the user to select values and settings used to control
the propagation of trajectories by the locbif algorithms.
\item[Description:] The locbif Orbit Panel is opened by selecting the {\bf Orbit Panel}
button located on the locbif Main Panel.  If unpinned, the panel vanishes when any button is selected.

\item[Panel items:]\mbox{}
	\begin{itemize}
	   \item {\bf itmap: } Iteration number.  (default: 1)
	   \item {\bf iorbit: }  {\em (Not used in current version)}
	   \item {\bf irhs: } Indicates the number of algebraic equations in the definition of a
	     differential-algebraic system.  Positive value indicates that the first {\em irhs} equations
	     in the dynamical system definition file are algebraic;  a negative value indicates the last
	     {\em irhs} equations are algebraic.  This parameter is meaningful only if the RADAU5 integrator
	     has been selected (see Solver selection item below). (default: 0)
	   \item {\bf isec: } If the Limit Cycle continuation mode has been selected (see Mode button 
	     description on locbif Main Panel above), this parameter is used to fix the phase of the
	     computed limit cycle.  If {\bf isec} is positive,  it indicates the index of the system definition
	     file equations used to specify a surface of section.  If {\bf isec} is negative, it indicates
	     the index of a user-defined auxilary function used to defined the surface of section.  If 
	     {\bf isec} is zero, a secant plane will be initially chosen and traced automatically to 
	     ensure it remains transverse to the limit cycle.  (default: 1)
	   \item {\bf dhint: } Stepsize used to numerically compute the Jacobian matrix for the equations
	     defining the vector field.  Used only for the RADAU5 integrator has been selected (see Solver
	     selection item below).  
	   \item {\bf epsint: } Absolute error tolerance for integration step. (default: 1.0e-6)
	   \item {\bf epsrel: } Relative error tolerance for integration step. (default: 1.0e-9)
	   \item {\bf h0int: } Initial stepsize for integration. (default: 0.1)
	   \item {\bf hmxint: } Maximum allowed stepsize for integration.  (default: 1.0)
	   \item {\bf solver: } Integration method selection item.  (default: 1)
	   \item {\bf tint: } Length of the time interval used for integration of vector fields.
	     Each initial value problem is presumed to be solved on the interval $[0,I_{tmap}*T_{int}]$.
	     (default: 10.0)
        \end{itemize}
\end{description}

\newpage
\subsection{locbif Display Options Panel}

\begin{figure}[tpbh]
\begin{picture}(400,170)(0,0)
%\put(135,0){\sunbitmap{3.0in}{3.0in}{fig4.ras}{75} }
\put(135,0){\psfig{figure=fig4.ps,height=2.7in,width=2.7in}}
\end{picture}
\caption{ \label{lbdis} Control options for the display of points computed in locbif.}
\end{figure}

\begin{description}

\item[Window title:] locbif Display Panel
\item[Function:]  This panel allows the user to select values and settings used to control
the display of points computed by the locbif continuation package.
\item[Description:] The locbif Display Panel is opened by selecting the {\bf Orbit Display}
button located on the locbif Main Panel.  If unpinned, the panel vanishes when any button is selected.

\item[Panel items:]\mbox{}
	\begin{itemize}
   \item {\bf soldot: } 
   \item {\bf isound: } If {\em On} is selected,  dstool will produce a tone each time a new point
	 is computed by locbif.  {\em (Not implemented)}
   \item {\bf iflash: } Parameter used in the locbif package to control the flashing of the cursor. 
	 This parameter is obsolete in the dstool version of the locbif interface.
   \item {\bf messag: } If {\em On} is selected, intermediate messages will be printed to the terminal
	 window used to initiate the dstool execution and printed in the Status message field on the 
	 locbif State panel.  If set to {\em Off}, no intermediate messages are produced.
	 (default: On)
   \item {\bf maxnpt: } Maximum number of points computed by locbif before returning control to the
	 dstool interface.  (default: 500)
   \item {\bf init: } This parameter is used by locbif to automatically assign values for the phase
	 space, parameter and system control settings.  The normal action of this parameter in obsolete
	 in the dstool version.  
        \end{itemize}
\end{description}



\newpage
\subsection{locbif State Panel}

\begin{figure}[tpbh]
\begin{picture}(400,150)(0,0)
%\put(20,0){\sunbitmap{3.0in}{3.0in}{fig5.ras}{55} }
\put(20,0){\psfig{figure=fig5.ps,height=2.3in,width=2.3in}}
\end{picture}
\caption{ \label{lbstate} The locbif State Panel is used to display data regarding the current point
computed by locbif.}
\end{figure}

\begin{description}

\item[Window title:] locbif State Panel
\item[Function:]  This panel displays the values of the phase space variables, dynamical system parameters,
user-defined functions and eigenvalues or Floquet multipliers for the current point computed by locbif.
\item[Description:] The locbif State Panel is opened by selecting the {\bf Cur State}
button located on the locbif Main Panel.  If unpinned, the panel vanishes when any button is selected.

\item[Panel items:]\mbox{}
	\begin{itemize}
	   \item {\bf Data:} The Data choice item is used to select the data to be displayed in the locbif
	   State Panel.  Four categories of data are avaliable for viewing: phase space variables,
	   dynamical system parameters, user-defined functions and eigenvalues or Floquet multipliers.
	   Any or all of these data types may be chosen in any combination, and the State Panel format
	   will automatically change to accomodate the user choices.  The fields used to display the 
	   data are read-only and updated each time program control is passed back to the user.
	   (default: Off, Off, Off, Off)
           \item {\bf Update Sel Pt:} This button may be used to update the dstool Selected Point
	   Window with phase space state and parameter data computed by locbif and displayed on the
	   locbif State Panel.  
	   \item {\bf Status: } Intermediate status message elicited by the locbif code are displayed
	   in this read-only field. 
	   \item {Phase Space Data Column:} Values for the phase space coordinates computed for the 
	   current locbif iteration.  Notice that the dstool Selected Point Window is not automatically
	   updated with the current values.  Thus, the Forward button on the locbif Main Panel may 
	   be used to restart the continuation segment at the original initial phase space state.
	   If the Continue button on the locbif Main Panel is depressed, pathfollowing will be 
	   initiated from the phase space coordinates displayed on the State Panel.  Thus,  {em two}
	   notions of an initial point are maintained by the program.
	   \item {Parameter Data Column:} Values for the dynamical system parameters computed for the 
	   current locbif iteration.  Notice that the dstool Selected Point Window is not automatically
	   updated with the current values.  Thus, the Forward button on the locbif Main Panel may 
	   be used to restart the continuation segment using the original parameter values.
	   If the Continue button on the locbif Main Panel is depressed, pathfollowing will be 
	   initiated from the parameter coordinates displayed on the State Panel.  Thus,  {em two}
	   notions of an initial point are maintained by the program.
	   \item {Function Data Column: } Values for the user-defined auxilary functions evaluated
	   for the phase space state and parameter values computed in the current locbif iteration.a
	   \item {Auxilary Data Column: } This column displays descriptive data for the current locbif
	   point and is dependent upon the type of pathfollowing problem selected on the Mode choice
	   item (see locbif Main Panel).  If continuation of Equilibria is selected, this column displays
	   the eigenvalues for the fixed point in the format $(a + b i)$.  For continuation of limit
	   cycles for autonomous vector fields is chosen, this column presents the Floquet multipliers
	   in the form $(modulus, argument)$.
        \end{itemize}
\end{description}


\clearpage

\begin{thebibliography}{99}

\bibitem{back1} Back, A., Guckenheimer, G., Myers, M., Wicklin, F. and
P. Worfolk, `dstool: Computer Assisted Exploration of Dynamical Systems,'
{\em AMS Notices}, April, 1992.

\bibitem{khibnik1} Khibnik, A., `LINLBF: A Program for Continuation and
Bifurcation Analysis of Equilibria Up to Codimension Three,'
{\em Continuation and Bifurcation: Numerical Techniques and Applications},
Edited by D. Roose et al., Kluwer Academic Publishers, Netherlands, 
1990, pps. 283-296.

\bibitem{khibnik2} Khibnik, A., Kuznetsov, Y., Levitin, V. and E. Nikolaev,
`LOCBIF: Interactive LOCal BIFurcation Analyzer', Users Guide, Version 2.1,
{\em in preparation}.

\end{thebibliography}
 


\end{document}
