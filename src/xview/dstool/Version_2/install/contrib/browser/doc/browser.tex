\documentstyle[psfig]{article}


% Definition for bitmap images to be included into LaTeX documents
%\def\sunbitmap#1#2#3#4{%
% \hfill \hbox to #1{%
%  \vbox to #2{
%   \vss
%   \special{pre: sunbitmap #3 #40}
%   \special{ps: sunbitmap #3 #40}
%  }%
% }%
% \hfill
%}

% set margins to 1 inch + specified quantity
\oddsidemargin 0.25in
\topmargin 0.25in
\headheight 0.25in

\textwidth 6in 
\headsep 0in
\textheight 8.25in

% set special fonts
\font\setfont=eufm10

\begin{document}


\begin{center}
{\LARGE \bf Memory Object Browser for dstool}  \\
\vspace{.15in}
{\large \bf Custom User Application Panel}
\end{center}

\vspace{.2in}
\begin{tabbing}
00000000000000000000\=00000000012345678\=00000123\=0123456789\=0123456789\=0123456789\=0123456789\=0123456789 \kill
\>{\bf Author: } \> Patrick A. Worfolk \\
\>	       \> Center for Applied Mathematics  \\
\>	       \> Cornell University \\
\>	       \> Ithaca, New York \hspace{.1in}  14853 \\
\>	       \> paw@macomb.tn.cornell.edu \\
\>	       \>   \\
\>{\bf Version: }\> 1.0 \hspace{.07in} for dstool Version 1.1   \\
\>	       \>   \\
\>{\bf Bug Reports: } \> dstool\_bugs@macomb.tn.cornell.edu
\end{tabbing}
\vspace{.2in}


\noindent {\large \bf Purpose:}  The Browser panel provides a convenient way of
	examining the points computed and stored during the execution of
	dstool.  Points from any memory objects may be selected and highlighted
	in the 2-D view windows, copied to the Selected Point window or deleted
	from storage.

\vspace{.15in}

\section{Overview}

\noindent One important task of the dstool program is to organize and store the 
(often copious amounts of) data generated by the computation routines during the
examination of a dynamical system.  These data are grouped into categories of
different types (trajectory, fixed points, etc.) and stored as individual units
called {\em memory objects}.  Within each memory object,  the stored data are
further organized into {\em flows}, composed of multiple {\em trajectories}
which represent collections of closely related data {\em points}.
These definitions arise as a result of the most important example of a memory
object -- that which stores the orbit data obtained by integrating or iterating
the dynamical system -- but its interpretation in dstool is much broader.
For more information regarding the 
organization of memory objects, see the dstool Reference Manual \cite{gucken1}.
\medskip

\noindent Although dstool handles the storage
and graphical display of the points in the memory objects automatically, it
is frequently useful for the user to peruse this database and examine the actual
values associated with particular stored data.  The Memory Object Browser
provides the dstool user with a flexible tool for inspecting the contents of a
prescribed memory object by displaying all the components which comprise a single
selected data point.  The viewed memory object and current selected point
may be changed at any time by panel choice items, allowing the user to freely 
sample data stored in different memory objects, or parts of the same object, as
required.  Moreover, the panel may be used to delete individual flows from a memory object
which typically represent collections of data stored by dstool as a single set. 
\medskip

\noindent One basic service that the Browser panel provides is the ability to 
restart a computation at a previously calculated point.  Once a data point is 
chosen for display on the Browser panel, it may be used to update relevant entries
on the Selected Point panel.  In addition, to facilitate the 
identification of individual points, a selected point in the Browser panel may be
highlighed in the graphical displays with crosshairs which appear only while the
cursor remains on the browser window.


\newpage
\section{The Browser Panel} \index{browser window}\index{window, browser}
%
%
%
\begin{figure}[h]
\vspace{.5in}
%\sunbitmap{3in}{2.1in}{browser.ras}{55}
\centerline{\psfig{figure=browser_fig.ps,height=3.0in,width=2.1in}}
\caption{\label{Browser}
The Browser Panel.
}
\end{figure}

%
%
%
\begin{description}

\item[Window title:] Data Browser
\item[Function:] The Browser panel allows the user to examine data points
	stored in the memory objects.  A point may be copied to the Selected
	Window or written to the standard output device.
\item[Description:] The Browser Panel is a user window and is optionally
	a part of the dstool executable.  It is opened by selecting the
	Browser option from the Panels menu button located on the Command
	Window.  
\item[Window type:] pop-up    
\item[Window attributes:] non-resizable, pinnable(in), variable template
\item[Panel items:]\mbox{}
	\begin{itemize}
	\item Memory stack setting: Allows the user to select a memory object
		for inspection.
	\item Flow numeric text field: Allows the user to select a flow
		from within the memory object.  The valid range of flow
		numbers within the selected memory object is displayed
		to the right of this field.
	\item Trajectory numeric text field:  Allows the user to select
		a particular trajectory from the flow.  The range
		of trajectories in the current flow object is displayed
		to the right of this field.
	\item Point numeric text field: Allows the user to select a point
		from the prescribed trajectory.  The range of points in the 
		trajectory is displayed to the right of this field.
	\item DELETE flow command button: Selection of this button will delete the 
		selected flow.  Dstool will NOT ask for confirmation.
		The view windows are not automatically updated, but the
		stored points field on the Command Window is adjusted.
	\item Write to stdout command button: Selection of this button
		writes the currently displayed point to the standard output device
		(normally the terminal).
	\item Highlight setting: Selcting the {\em ON} option will cause the representation
		of the current point in the view windows to be identified by the
		centering of a pair of crosshairs on the point in each graphical window.
		These crosshairs will disappear when the mouse pointer leaves the browser window.
	\item Copy to Selected command button: Selection of this button
		will copy the relevant components of the current point into the Selected Points
		panel.
	\end{itemize}
\item[Library name:] browserlib.a
\item[User panel interface:] \mbox{}
	\begin{verbatim}
	extern Menu_item browser_handler();
	extern int browser_field_manager(); 
	\end{verbatim}
\item[User panel entry:] \begin{verbatim}
	 {"Browser...", browser_handler, browser_field_manager}
	\end{verbatim}

\end{description}


\begin{thebibliography}{99}

\bibitem{gucken1} J. Guckenheimer, M. Myers, F. Wicklin and P. Worfolk, 
`dstool Reference Manual,' Center for Applied Mathematics, Cornell University,
1992.

\end{thebibliography}


\end{document}


