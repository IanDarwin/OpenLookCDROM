\documentstyle[fullpage]{article}

\begin{document}

\title{Weekly Status Report}

\author{Group 7 --- Project Planning Tool\\
Project Manager: Brian Gaubert\\
Project Coordinator: Mike Marlow\\
Documentation Manager: Mark M. Lacey\\
Configuration Manager: Richard Malingkas}

\date{Due October 26, 1992}

\maketitle

%
% This section is used to list the accomplishments of the week.
% Use the \item construct to list each item.  Try to keep the
% Descriptions for each down to one or two sentences
%
% Limit your accomplishments to items specific to the progress of the project.
%
\section{Planned Accomplishments}
\begin{itemize}
	\item Sunday, we talked about the interface and integration
between the different modules. We also talked about the file
read/write, database, and the calculations in a very descriptive
manner. We discussed about how the pointers will be assigned for the
tasks within our database struct and how they will be referenced from
other modules. (3 hours)

\end{itemize}

%
% This section is used to list the unscheduled accomplishments of the week.
% Use the \item construct to list each item.  Try to keep the
% Descriptions for each down to one or two sentences
%
% Limit your accomplishments to items specific to the progress of the project.
%
\section{Other Accomplishments}
\begin{itemize}
	\item Defined a structure for the task list and its necessities.
\end{itemize}

%
% This section is used to list the following weeks plan
% Use the \item construct to list each item.  Try to keep the
% Descriptions for each down to one or two sentences
%
% Limit your plans to items specific to the progress of the project.
%
\section{Next Weeks Plan}
\begin{itemize}
	\item Proof-read and integrate the detailed design for Wednesday.
\end{itemize}

%
% This section is used to list any issues that were raised during
% the week that are of special interest.  Also use this to voice
% issues that the TA or Instructor must resolve.
% Use the \item construct to list each item.  Try to keep the
% Descriptions for each down to one or two sentences
%
% Limit your issues to items specific to the progress of the project.
%
\section{Issues}
\begin{itemize}
	\item There are no issues at this time.
\end{itemize}

\end{document}


